\PassOptionsToPackage{unicode=true}{hyperref} % options for packages loaded elsewhere
\PassOptionsToPackage{hyphens}{url}
%
\documentclass[]{article}
\usepackage{lmodern}
\usepackage{amssymb,amsmath}
\usepackage{ifxetex,ifluatex}
\usepackage{fixltx2e} % provides \textsubscript
\ifnum 0\ifxetex 1\fi\ifluatex 1\fi=0 % if pdftex
  \usepackage[T1]{fontenc}
  \usepackage[utf8]{inputenc}
  \usepackage{textcomp} % provides euro and other symbols
\else % if luatex or xelatex
  \usepackage{unicode-math}
  \defaultfontfeatures{Ligatures=TeX,Scale=MatchLowercase}
\fi
% use upquote if available, for straight quotes in verbatim environments
\IfFileExists{upquote.sty}{\usepackage{upquote}}{}
% use microtype if available
\IfFileExists{microtype.sty}{%
\usepackage[]{microtype}
\UseMicrotypeSet[protrusion]{basicmath} % disable protrusion for tt fonts
}{}
\IfFileExists{parskip.sty}{%
\usepackage{parskip}
}{% else
\setlength{\parindent}{0pt}
\setlength{\parskip}{6pt plus 2pt minus 1pt}
}
\usepackage{hyperref}
\hypersetup{
            pdftitle={Comparison between GxEScanR and GLM for 2df/3df models},
            pdfborder={0 0 0},
            breaklinks=true}
\urlstyle{same}  % don't use monospace font for urls
\usepackage[margin=0.5in]{geometry}
\usepackage{graphicx,grffile}
\makeatletter
\def\maxwidth{\ifdim\Gin@nat@width>\linewidth\linewidth\else\Gin@nat@width\fi}
\def\maxheight{\ifdim\Gin@nat@height>\textheight\textheight\else\Gin@nat@height\fi}
\makeatother
% Scale images if necessary, so that they will not overflow the page
% margins by default, and it is still possible to overwrite the defaults
% using explicit options in \includegraphics[width, height, ...]{}
\setkeys{Gin}{width=\maxwidth,height=\maxheight,keepaspectratio}
\setlength{\emergencystretch}{3em}  % prevent overfull lines
\providecommand{\tightlist}{%
  \setlength{\itemsep}{0pt}\setlength{\parskip}{0pt}}
\setcounter{secnumdepth}{0}
% Redefines (sub)paragraphs to behave more like sections
\ifx\paragraph\undefined\else
\let\oldparagraph\paragraph
\renewcommand{\paragraph}[1]{\oldparagraph{#1}\mbox{}}
\fi
\ifx\subparagraph\undefined\else
\let\oldsubparagraph\subparagraph
\renewcommand{\subparagraph}[1]{\oldsubparagraph{#1}\mbox{}}
\fi

% set default figure placement to htbp
\makeatletter
\def\fps@figure{htbp}
\makeatother

\usepackage{etoolbox}
\makeatletter
\providecommand{\subtitle}[1]{% add subtitle to \maketitle
  \apptocmd{\@title}{\par {\large #1 \par}}{}{}
}
\makeatother
% https://github.com/rstudio/rmarkdown/issues/337
\let\rmarkdownfootnote\footnote%
\def\footnote{\protect\rmarkdownfootnote}

% https://github.com/rstudio/rmarkdown/pull/252
\usepackage{titling}
\setlength{\droptitle}{-2em}

\pretitle{\vspace{\droptitle}\centering\huge}
\posttitle{\par}

\preauthor{\centering\large\emph}
\postauthor{\par}

\predate{\centering\large\emph}
\postdate{\par}
\usepackage{booktabs}
\usepackage{longtable}
\usepackage{array}
\usepackage{multirow}
\usepackage{wrapfig}
\usepackage{float}
\usepackage{colortbl}
\usepackage{pdflscape}
\usepackage{tabu}
\usepackage{threeparttable}
\usepackage{threeparttablex}
\usepackage[normalem]{ulem}
\usepackage{makecell}
\usepackage{xcolor}

\title{Comparison between GxEScanR and GLM for 2df/3df models}
\date{Updated on 2020-04-07}

\begin{document}
\maketitle

\hypertarget{goal}{%
\section{Goal}\label{goal}}

For post-hoc analyses, need to validate hits by fitting various models
(GxE, joint models) with additional covariates. In the process, noticed
that 3DF results have very small differences between GLM and GxEScanR

In here, I'm summarizing 2DF/3DF results for 10 SNPs from the Type II
Diabetes GWIS, for both GxEScanR and GLM

\hypertarget{models-for-likelihood-ratio-test}{%
\subsection{Models for likelihood ratio
test}\label{models-for-likelihood-ratio-test}}

\hypertarget{df-joint-test}{%
\subsubsection{2DF joint test}\label{df-joint-test}}

\begin{itemize}
\tightlist
\item
  base model:
  \(Outcome \sim SNP + E + age + sex + pc1 + pc2 + pc3 + study\)
\item
  interaction model:
  \(Outcome \sim SNP + E + SNP*E + age + sex + pc1 + pc2 + pc3 + study\)
\end{itemize}

\hypertarget{df-joint-test-1}{%
\subsubsection{3DF joint test}\label{df-joint-test-1}}

\begin{itemize}
\tightlist
\item
  base model: \(SNP \sim age + sex + pc1 + pc2 + pc3 + study\)
\item
  interaction model:
  \(SNP \sim Outcome + E + Outcome*E + age + sex + pc1 + pc2 + pc3 + study\)
\end{itemize}

\hypertarget{p-values}{%
\section{P-values}\label{p-values}}

\resizebox{\linewidth}{!}{
\begin{tabular}{l|r|l|l|l|l}
\hline
SNP & Subjects & pval\_gxescan\_2df & pval\_glm\_2df & pval\_gxescan\_3df & pval\_glm\_3df\\
\hline
1:71040166:G:T & 74390 & 1.77074e-07 & 1.77072e-07 & 1.25551e-09 & 1.57517e-09\\
\hline
10:114754071:T:C & 74390 & 6.01701e-01 & 6.01701e-01 & 2.45775e-49 & 1.89116e-49\\
\hline
10:114784926:C:T & 74390 & 2.82608e-01 & 2.82608e-01 & 4.83087e-08 & 4.21936e-08\\
\hline
12:4384844:T:G & 74390 & 6.61997e-04 & 6.61980e-04 & 4.42451e-10 & 6.31504e-10\\
\hline
16:53811788:A:G & 74390 & 9.87860e-01 & 9.87860e-01 & 4.72788e-08 & 4.76705e-08\\
\hline
20:6442961:G:A & 74390 & 2.67526e-07 & 2.67523e-07 & 3.89925e-08 & 5.62870e-08\\
\hline
3:185510884:A:C & 74390 & 2.57710e-01 & 2.57710e-01 & 1.02726e-11 & 1.31501e-11\\
\hline
6:20688121:T:A & 74390 & 3.75289e-05 & 3.75294e-05 & 2.75143e-11 & 2.28556e-11\\
\hline
7:28189411:T:C & 74390 & 4.66363e-01 & 4.66362e-01 & 5.74195e-10 & 4.32742e-10\\
\hline
8:118185025:G:A & 74390 & 2.75634e-03 & 2.75633e-03 & 7.87174e-11 & 6.06649e-10\\
\hline
\end{tabular}}

\hypertarget{chi-square-values}{%
\section{chi-square values}\label{chi-square-values}}

\resizebox{\linewidth}{!}{
\begin{tabular}{l|r|r|r|r|r|r}
\hline
SNP & Subjects & chiSqGE & chiSq2df & chiSq\_glm\_2df & chiSq3df & chiSq\_glm\_3df\\
\hline
1:71040166:G:T & 74390 & 13.28280 & 31.0934000 & 31.0934149 & 44.37620 & 43.91250\\
\hline
10:114754071:T:C & 74390 & 227.82900 & 1.0159900 & 1.0159894 & 228.84499 & 229.37138\\
\hline
10:114784926:C:T & 74390 & 34.37270 & 2.5273900 & 2.5273899 & 36.90009 & 37.17790\\
\hline
12:4384844:T:G & 74390 & 31.86660 & 14.6405000 & 14.6405494 & 46.50710 & 45.78045\\
\hline
16:53811788:A:G & 74390 & 36.91990 & 0.0244291 & 0.0244291 & 36.94433 & 36.92739\\
\hline
20:6442961:G:A & 74390 & 7.07173 & 30.2681000 & 30.2681184 & 37.33983 & 36.58628\\
\hline
3:185510884:A:C & 74390 & 51.46790 & 2.7118400 & 2.7118440 & 54.17974 & 53.67686\\
\hline
6:20688121:T:A & 74390 & 31.79210 & 20.3808000 & 20.3807719 & 52.17290 & 52.55087\\
\hline
7:28189411:T:C & 74390 & 44.44920 & 1.5255800 & 1.5255850 & 45.97478 & 46.55241\\
\hline
8:118185025:G:A & 74390 & 38.24250 & 11.7877000 & 11.7877140 & 50.03020 & 45.86247\\
\hline
\end{tabular}}

You can see results for 2DF are essentially the same, while there are
slight differences for 3DF. What do you think? Is this small difference
acceptable?

\hypertarget{fixed}{%
\section{Fixed}\label{fixed}}

Using GLM I'll simply calculate 3DF pvalues the same exact way as
gxescan e.g.~adding E\textbar{}G (linear) to 2DF chiSq and calculalate
3df p values. Once i do that the chiSq 3DF values are consistent:

\resizebox{\linewidth}{!}{
\begin{tabular}{l|r|r|r|r|r|r}
\hline
SNP & Subjects & chiSqGE & chiSq2df & chiSq\_glm\_2df & chiSq3df & chiSq\_glm\_3df\\
\hline
1:71040166:G:T & 74390 & 13.28280 & 31.0934000 & 31.0934149 & 44.37620 & 44.37617\\
\hline
10:114754071:T:C & 74390 & 227.82900 & 1.0159900 & 1.0159894 & 228.84499 & 228.84546\\
\hline
10:114784926:C:T & 74390 & 34.37270 & 2.5273900 & 2.5273899 & 36.90009 & 36.90013\\
\hline
12:4384844:T:G & 74390 & 31.86660 & 14.6405000 & 14.6405494 & 46.50710 & 46.50714\\
\hline
16:53811788:A:G & 74390 & 36.91990 & 0.0244291 & 0.0244291 & 36.94433 & 36.94436\\
\hline
20:6442961:G:A & 74390 & 7.07173 & 30.2681000 & 30.2681184 & 37.33983 & 37.33984\\
\hline
3:185510884:A:C & 74390 & 51.46790 & 2.7118400 & 2.7118440 & 54.17974 & 54.17970\\
\hline
6:20688121:T:A & 74390 & 31.79210 & 20.3808000 & 20.3807719 & 52.17290 & 52.17289\\
\hline
7:28189411:T:C & 74390 & 44.44920 & 1.5255800 & 1.5255850 & 45.97478 & 45.97475\\
\hline
8:118185025:G:A & 74390 & 38.24250 & 11.7877000 & 11.7877140 & 50.03020 & 50.03022\\
\hline
\end{tabular}}

\end{document}
